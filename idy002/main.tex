\documentclass[a4paper,openany]{book}
%\usepackage{ctex}
\usepackage{bm}
\usepackage{longtable}
%\usepackage[fleqn]{amsmath}
\usepackage{harpoon}
 \usepackage{fontspec}
\usepackage{listings}
\usepackage[left=2.5cm, right=2.5cm, top=2.5cm, bottom=2.5cm]{geometry}
\usepackage{setspace}
\usepackage{bm}
\usepackage{cmap}
\usepackage{cite}
\usepackage{float}
\usepackage{xeCJK}
\usepackage{amsthm}
\usepackage{amsmath}
\usepackage{amssymb}
\usepackage{setspace}
\usepackage{enumerate}
\usepackage{indentfirst}

\usepackage[table]{xcolor}
\usepackage{booktabs}
\usepackage[cache=false]{minted}
\usepackage{pdfpages}
\allowdisplaybreaks

%\setlength{\parindent}{0em}
%\setlength{\mathindent}{0pt}

\newfontfamily\Courier{Courier 10 Pitch}
\renewcommand{\theFancyVerbLine}{\rmfamily\scriptsize\arabic{FancyVerbLine}}
\newcommand{\cppcode}[1]{
    \inputminted[mathescape,
    			 tabsize=2,
    			 linenos,
    			 %frame=single,
    			 framesep=2mm,
    			 breakaftergroup=true,
    			 breakautoindent=true,
    			 breakbytoken=true,
    			 breaklines=true,
    			 fontsize=\small
    ]{cpp}{source/#1}
}
\newcommand{\javacode}[1]{
    \inputminted[mathescape,
    			 tabsize=2,
    			 linenos,
    			 %frame=single,
    			 framesep=2mm,
    			 breakaftergroup=true,
    			 breakautoindent=true,
    			 breakbytoken=true,
    			 breaklines=true,
    			 fontsize=\small
    ]{java}{source/#1}
}
\newcommand{\vimcode}[1]{
    \inputminted[mathescape,
    			 tabsize=2,
    			 linenos,
    			 %frame=single,
    			 framesep=2mm,
    			 breakaftergroup=true,
    			 breakautoindent=true,
    			 breakbytoken=true,
    			 breaklines=true,
    			 fontsize=\small
    ]{vim}{source/#1}
}
\begin{document}	
	\title{\textbf{\LARGE{Standard Code Library}}}
	\author{Shanghai Jiao Tong University}
	\date{\today}
	\maketitle
	\tableofcontents
	\begin{spacing}{0.9}

	\chapter{图论}
		\section{强连通分量}
		\section{双连通分量}
		\section{Dinic网络流}
		\section{费用流}
		\section{kth短路}
		\section{Two-SAT}
		\section{KM最大权匹配}
	\chapter{数论}
	\chapter{组合}
	\chapter{字符串}
	\chapter{计算几何}
	\chapter{数据结构}
	\chapter{其他}
		\section{Java Hints}
		\javacode{Hint/Java-Hints.java}
		\section{vimrc}
		\vimcode{Hint/vimrc}
		\section{常用结论}
			\subsection{上下界网络流}
			$B(u,v)$表示边$(u,v)$流量的下界,$C(u,v)$表示边$(u,v)$流量的上界,$F(u,v)$表示边$(u,v)$的流量。
			设$G(u,v) = F(u,v) - B(u,v)$,显然有
			$$0 \leq G(u,v) \leq C(u,v)-B(u,v)$$
			\subsubsection*{无源汇的上下界可行流}
			建立超级源点$S^*$和超级汇点$T^*$,对于原图每条边$(u,v)$在新网络中连如下三条边:$S^* \rightarrow v$,容量为$B(u,v)$;$u \rightarrow T^*$,容量为$B(u,v)$;$u \rightarrow v$,容量为$C(u,v) - B(u,v)$。最后求新网络的最大流,判断从超级源点$S^*$出发的边是否都满流即可,边$(u,v)$的最终解中的实际流量为$G(u,v)+B(u,v)$。
			\subsubsection*{有源汇的上下界可行流}
			从汇点$T$到源点$S$连一条上界为$\infty$,下界为$0$的边。按照\textbf{无源汇的上下界可行流}一样做即可,流量即为$T \rightarrow S$边上的流量。
			\subsubsection*{有源汇的上下界最大流}
			\begin{enumerate}
				\item 在\textbf{有源汇的上下界可行流}中,从汇点$T$到源点$S$的边改为连一条上界为$\infty$,下届为$x$的边。$x$满足二分性质,找到最大的$x$使得新网络存在\textbf{无源汇的上下界可行流}即为原图的最大流。
				\item 从汇点$T$到源点$S$连一条上界为$\infty$,下界为$0$的边,变成无源汇的网络。按照\textbf{无源汇的上下界可行流}的方法,建立超级源点$S^*$和超级汇点$T^*$,求一遍$S^* \rightarrow T^*$的最大流,再将从汇点$T$到源点$S$的这条边拆掉,求一次$S \rightarrow T$的最大流即可。
			\end{enumerate}
			\subsubsection*{有源汇的上下界最小流}
			\begin{enumerate}
				\item 在\textbf{有源汇的上下界可行流}中,从汇点$T$到源点$S$的边改为连一条上界为$x$,下界为$0$的边。$x$满足二分性质,找到最小的$x$使得新网络存在\textbf{无源汇的上下界可行流}即为原图的最小流。
				\item 按照\textbf{无源汇的上下界可行流}的方法,建立超级源点$S^*$与超级汇点$T^*$,求一遍$S^* \rightarrow T^*$的最大流,但是注意这一次不加上汇点$T$到源点$S$的这条边,即不使之改为无源汇的网络去求解。求完后,再加上那条汇点$T$到源点$S$上界$\infty$的边。因为这条边下界为$0$,所以$S^*$,$T^*$无影响,再直接求一次$S^* \rightarrow T^*$的最大流。若超级源点$S^*$出发的边全部满流,则$T \rightarrow S$边上的流量即为原图的最小流,否则无解。
			\end{enumerate}
			\subsection*{上下界费用流}
			\noindent \textbf{来源:BZOJ 3876}
			\noindent 设汇$t$,源$s$,超级源$S$,超级汇$T$,本质是每条边的下界为1,上界为MAX,跑一遍有源汇的上下界最小费用最小流。(因为上界无穷大,所以只要满足所有下界的最小费用最小流)
			\begin{enumerate}
				\item 对每个点$x$:从$x$到$t$连一条费用为0,流量为MAX的边,表示可以任意停止当前的剧情(接下来的剧情从更优的路径去走,画个样例就知道了)
				\item 对于每一条边权为z的边x->y:
				\begin{itemize}
					\item 从S到y连一条流量为1,费用为z的边,代表这条边至少要被走一次。
					\item 从x到y连一条流量为MAX,费用为z的边,代表这条边除了至少走的一次之外还可以随便走。
					\item 从x到T连一条流量为1,费用为0的边。(注意是每一条x->y的边都连,或者你可以记下x的出边数Kx,连一次流量为Kx,费用为0的边)。
				\end{itemize}
			\end{enumerate}
			建完图后从S到T跑一遍费用流,即可。(当前跑出来的就是满足上下界的最小费用最小流了)
			\subsection{弦图相关}
			\begin{enumerate}
				\item[1.] 团数 $\leq$ 色数 , 弦图团数 = 色数
				\item[2.] 设 $next(v)$ 表示 $N(v)$ 中最前的点 . 
				令 w* 表示所有满足 $A \in B$ 的 w 中最后的一个点 , 
				判断 $v \cup N(v)$ 是否为极大团 , 
				只需判断是否存在一个 w, 
				满足 $Next(w)=v$ 且 $|N(v)| + 1 \leq |N(w)|$ 即可 . 
				\item[3.] 最小染色 : 完美消除序列从后往前依次给每个点染色 , 
				给每个点染上可以染的最小的颜色
				\item[4.] 最大独立集 : 完美消除序列从前往后能选就选
				\item[5.] 弦图最大独立集数 $=$ 最小团覆盖数 , 
				最小团覆盖 : 
				设最大独立集为 $\{p_1,p_2, \dots ,p_t\}$, 
				则 $\{p_1\cup N(p_1), \dots , p_t \cup N(p_t)\}$ 
				为最小团覆盖
			\end{enumerate}
			\subsection{Bernoulli数}
				\input{source/Hint/Bernoulli-Number.tex}
		\section{常见错误}
		\begin{spacing}{0.6}
		\begin{enumerate}
			\item 数组或者变量类型开错,例如将double开成int;
			\item 函数忘记返回返回值;
			\item 初始化数组没有初始化完全;
			\item 对空间限制判断不足导致MLE;
		\end{enumerate}
		\end{spacing}
		\section{测试列表}
		\begin{spacing}{0.6}
		\begin{enumerate}
			\item 检测评测机是否开O2;
			\item 检测\_\_int128以及\_\_float128是否能够使用;
			\item 检测是否能够使用C++11;
			\item 检测是否能够使用Ext Lib;
			\item 检测程序运行所能使用的内存大小;
			\item 检测程序运行所能使用的栈大小;
			\item 检测是否有代码长度限制;
			\item 检测是否能够正常返Runtime Error(assertion、return 1、空指针);
			\item 查清楚厕所方位和打印机方位;
		\end{enumerate}
		\end{spacing}
		\section{博弈游戏}
		\input{source/Hint/Game-Theory.tex}
		\section{常用数学公式}
		\input{source/Hint/Mathematical-Lemma.tex}
		\section{附录}
		\subsection{NTT素数及原根列表}
		\input{source/Appendix/NFT-Primes.tex}
		\includepdf[pages=-]{source/Appendix/cheat.pdf}
	\end{spacing}
\end{document}
